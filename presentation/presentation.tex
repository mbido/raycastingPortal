\documentclass{beamer}
\usepackage{caption}
\usepackage[utf8]{inputenc}
\usepackage[T1]{fontenc}


% Pour quel partie chaque personne parle :
% Mise en contexte -> Vital
% Technologies utilisées -> Vital
% Différentes stratégies pour un rendu 3D :
%   Approche historique -> Matthieu
%   Explication de DDA  -> Matthieu
%   Approche moderne -> Matthieu
% Construction de mur -> Vital
% Les collisions -> Elian
% Les portails -> Matthieu / Elian
% Les renders -> Matthieu
% Conclusion -> Elian


\usetheme{Boadilla}

\begin{document}

% Définition de la mise en page du pied de page avec un espacement réduit
\setbeamertemplate{footline}{%
  \leavevmode%
  \hbox{\begin{beamercolorbox}[wd=\paperwidth,ht=2.25ex,dp=1ex,center]{author in head/foot}%
    \usebeamerfont{author in head/foot}\insertshortauthor\hspace*{2em}
    \insertshorttitle\hspace*{2em}\insertframenumber{} / \inserttotalframenumber
  \end{beamercolorbox}}%
  \vskip0pt%
}

\title{Présentation du projet Portal 0.0}
\author[BADSTÜBER E. BIDAULT M. FOCHEUX V.]{BADSTÜBER Elian BIDAULT Matthieu FOCHEUX Vital \\
                                                Licence 3 Informatique}
\date{Mars 2024}

\setcounter{framenumber}{0}
% \setbeamertemplate{page number in head/foot}{}

{
\setbeamertemplate{footline}{}
\begin{frame}
    \titlepage
    
    \vfill % Insertion d'un espace vertical flexible pour pousser le texte en bas
    \begin{columns}
        \column{0.3\textwidth}
        \centering
        \includegraphics[width=\textwidth]{images/logo-UFR-ST.jpg} 
    
        \column{0.4\textwidth}
        \begin{flushright}
            \small Tuteur : Julien BERNARD
        \end{flushright}    
    \end{columns}
\end{frame}
}

\setcounter{framenumber}{0}
\setbeamertemplate{page number in head/foot}[totalframenumber]

\begin{frame}
    \frametitle{Table des matières}
    \tableofcontents
\end{frame}

\section{Mise en contexte}

\begin{frame}
    \frametitle{Mise en contexte}
    \begin{block}{}
        % Portal 0.0 est un jeu vidéo qui reprends les principes techniques 
        % de plusieurs jeux vidéos connus.
        \begin{itemize}
            \item Portal 0.0 $\rightarrow $ principes techniques de plusieurs jeux vidéos connus
        \end{itemize}
    \end{block}

    \begin{block}{Technique graphique}
        \begin{itemize}
            \item Méthode raycasting
            \item Rendu 2.5D popularisé dans les années 90
            \item Principe de Wolfenstein3D (1992)
        \end{itemize}
        % L'utilisation de la méthode du raycasting, pour un rendu 2.5D qui a été 
        % popularisé dans le début des années 90.
    \end{block}

    \begin{block}{Système de jeu}
        \begin{itemize}
            \item Résolution d'énigmes à l'aide de portails
            \item Téléportation lorsqu'on passe à travers
            \item Principe de Portal (2007)
        \end{itemize}
        % Le principe du jeu vidéo Portal (2007), où l'on doit résoudre des énigmes
        % à l'aide de portails, dans lesquels on se téléporte lorsqu'on passe à travers.
    \end{block}
\end{frame}

\section{Technologies utilisées}

\begin{frame}
    \frametitle{Technologies utilisées}
    \begin{columns}
        \column{0.3\textwidth}
        \centering
        \includegraphics[width=\textwidth]{images/cpp.png} 
    
        \column{0.4\textwidth}
        \centering
        \includegraphics[width=\textwidth]{images/GF.png}
        \captionof{figure}{Gamedev Framework}

        \column{0.4\textwidth}
        \centering
        \includegraphics[width=\textwidth]{images/github.png}
    \end{columns}
\end{frame}

\section{Détails du développement}
\subsection{Différentes stratégies pour un rendu 3D}

\begin{frame}
    \frametitle{Différentes stratégie pour un rendu 3D \\
                \small Approche historique}
    \begin{block}{Raycasting}
        \begin{itemize}
            \item Méthode de rendu graphique
            \item Création d'une perspective 3D dans des environnements 2D
            \item Réalisation grâce à une projection de rayons à travers la scène
        \end{itemize}
    \end{block}
    \begin{block}{}
        % Le raycasting est une méthode de rendu graphique utilisée pour créer 
        % une perspective 3D dans des environnements 2D. Ce rendu est réalisé
        % en projetant des rayons depuis la position du joueur à travser la scène
        % pour déterminer les intersections avec les murs. Pour l'optimisation,
        % l'algorithme DDA (Digital Differential Analyzer) est utilisé.
        Pour l'optimisation, l'algorithme DDA (Digital Differential Analyzer) est utilisé.
    \end{block}
\end{frame}

\begin{frame}
    \frametitle{Différentes stratégie pour un rendu 3D \\
                \small DDA}           
    \begin{block}{}
        \begin{itemize}
            \item Conçu pour la rasterisation de lignes
            \item Employé pour déterminer où les rayons projetés intersectent avec les objets de l'environnement
            \item Repose sur l'itération linéaire
        \end{itemize}
        % L'algorithme DDA est historiquement conçu pour la rasterisation de 
        % lignes, c'est-à-dire la conversion de lignes geometriques en ligne 
        % visuelle composées de pixels alignés. Dans le contexte du raycasting, 
        % DDA est employé pour déterminer où les rayons projetés à travers la 
        % scène intersectent avec les objets de l'environnement, typiquement et 
        % dans notre cas représenté par une grille de cellules. Le principe 
        % fondamental de DDA repose sur l'itération linéaire. Plutôt que de 
        % calculer chaque point le long d'une demie droite en utilisant des formules 
        % de géométrie directe, ce qui pourrait être coûteux en termes de performances, 
        % DDA avance par petits incr ́ements. Cela signifie que pour chaque
        % pas sur l'axe le plus dominant (x ou y), DDA calcule l'emplacement 
        % correspondant sur l'autre axe en ajoutant un incrément constant. 
        % Cela se traduit par un parcours régulier et efficace le long de la 
        % ligne en faisant des sauts à chaque intersection entre la demie 
        % droite et la grille 2D. Dans le raycasting, DDA est utilisé pour 
        % trouver rapidement et précisément les intersections entre les rayons 
        % et les murs
    \end{block}    
\end{frame}

\begin{frame}
    \frametitle{Différentes stratégie pour un rendu 3D \\
                \small DDA}           
    \begin{block}{}
        \begin{itemize}
            \item Calcule l'emplacement correspondant sur l'autre axe en ajoutant un incrément constant
            \item Utilisé pour trouver rapidement et précisément les intersections entre les rayons 
            et les murs dans le raycasting
        \end{itemize}
        % L'algorithme DDA est historiquement conçu pour la rasterisation de 
        % lignes, c'est-à-dire la conversion de lignes geometriques en ligne 
        % visuelle composées de pixels alignés. Dans le contexte du raycasting, 
        % DDA est employé pour déterminer où les rayons projetés à travers la 
        % scène intersectent avec les objets de l'environnement, typiquement et 
        % dans notre cas représenté par une grille de cellules. Le principe 
        % fondamental de DDA repose sur l'itération linéaire. Plutôt que de 
        % calculer chaque point le long d'une demie droite en utilisant des formules 
        % de géométrie directe, ce qui pourrait être coûteux en termes de performances, 
        % DDA avance par petits incr ́ements. Cela signifie que pour chaque
        % pas sur l'axe le plus dominant (x ou y), DDA calcule l'emplacement 
        % correspondant sur l'autre axe en ajoutant un incrément constant. 
        % Cela se traduit par un parcours régulier et efficace le long de la 
        % ligne en faisant des sauts à chaque intersection entre la demie 
        % droite et la grille 2D. Dans le raycasting, DDA est utilisé pour 
        % trouver rapidement et précisément les intersections entre les rayons 
        % et les murs
    \end{block}    
\end{frame}

\begin{frame}
    \frametitle{Différentes stratégie pour un rendu 3D \\
                \small Approche moderne}           
    \begin{block}{}
        
    \end{block}    
\end{frame}

\subsection{Construction de mur}

\begin{frame}
    \frametitle{Construction de mur}
    \begin{block}{}
    \end{block}
\end{frame}

\subsection{Les collisions}

\begin{frame}
    \frametitle{Les collisions}
    \begin{block}{}
    \end{block}
\end{frame}



\section{Spécifications}
\subsection{Les portails}

\begin{frame}
    \frametitle{Les portails}
    \begin{block}{}
        
    \end{block}
\end{frame}

\subsection{Les rendus}

\begin{frame}
    \frametitle{Les rendus}
    \begin{block}{}
        
    \end{block}
\end{frame}

\section{Conclusion}

\begin{frame}
    \frametitle{Conclusion}
    \begin{block}{}
        \centering
        Merci de votre attention.
    \end{block}
\end{frame}

\section*{Questions}

\begin{frame}
    \frametitle{Questions}
    \begin{block}{}
        \centering
        Avez-vous des questions ?
    \end{block}
\end{frame}

\end{document}