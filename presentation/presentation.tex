\documentclass{beamer}
\usepackage{caption}

% Pour quel partie chaque personne parle :
% Mise en contexte -> Vital
% Technologies utilisées -> Elian? Vital?
% Différentes stratégies pour un rendu 3D :
%   Approche historique -> Vital
%   Explication de DDA  -> Matthieu
%   Approche moderne -> Matthieu
% Construction de mur -> Vital
% Les collisions -> Elian
% Les portails -> Matthieu
% Les renders -> Matthieu
% Conclusion -> Elian? Vital?


\usetheme{Berlin}

\begin{document}

\title{Présentation du projet Portal 0.0}
\author{BADSTÜBER Elian, BIDAULT Matthieu, FOCHEUX Vital \\
        Licence 3 Informatique}
\date{\today}

\begin{frame}
    \titlepage

    \vfill % Insertion d'un espace vertical flexible pour pousser le texte en bas
    \begin{columns}
        \column{0.3\textwidth}
        \centering
        \includegraphics[width=\textwidth]{images/logo-UFR-ST.jpg} 
    
        \column{0.4\textwidth}
        \begin{flushright}
            \small Tuteur : Julien BERNARD
        \end{flushright}    
    \end{columns}
\end{frame}

\setcounter{framenumber}{0}
\setbeamertemplate{page number in head/foot}[totalframenumber]

\begin{frame}
    \frametitle{Table des matières}
    \tableofcontents
\end{frame}

\section{Mise en contexte}

\begin{frame}
    \frametitle{Mise en contexte}
    \begin{block}{}
        Portal 0.0 est un jeu vidéo qui reprends les principes techniques 
        de plusieurs jeux vidéos connus.
    \end{block}

    \begin{block}{Technique graphique}
        L'utilisation de la méthode du raycasting, pour un rendu 2.5D qui a été 
        popularisé dans le début des années 90.
    \end{block}

    \begin{block}{Système de jeu}
        Le principe du jeu vidéo Portal (2007), où l'on doit résoudre des énigmes
        à l'aide de portails, dans lesquels on se téléporte lorsqu'on passe à travers.
    \end{block}
\end{frame}

\section{Technologies utilisées}

\begin{frame}
    \frametitle{Technologies utilisées}
    \begin{columns}
        \column{0.3\textwidth}
        \centering
        \includegraphics[width=\textwidth]{images/cpp.png} 
    
        \column{0.4\textwidth}
        \centering
        \includegraphics[width=\textwidth]{images/GF.png}
        \captionof{figure}{Gamedev Framework}

        \column{0.4\textwidth}
        \centering
        \includegraphics[width=\textwidth]{images/github.png}
    \end{columns}
\end{frame}

\section{Détails du développement}
\subsection{Différentes stratégies pour un rendu 3D}

\begin{frame}
    \frametitle{Différentes stratégie pour un rendu 3D \\
                \small Approche historique}
    \begin{block}{}
        Le raycasting est une méthode de rendu graphique utilisée pour créer 
        une perspective 3D dans des environnements 2D. Ce rendu est réalisé
        en projetant des rayons depuis la position du joueur à travser la scène
        pour déterminer les intersections avec les murs. Pour l'optimisation,
        l'algorithme DDA (Digital Differential Analyzer) est utilisé.
    \end{block}
\end{frame}

\begin{frame}
    \frametitle{Différentes stratégie pour un rendu 3D \\
                \small DDA}           
    \begin{block}{}
        L'algorithme DDA est historiquement conçu pour la rasterisation de 
        lignes, c'est-à-dire la conversion de lignes geometriques en ligne 
        visuelle composées de pixels alignés. Dans le contexte du raycasting, 
        DDA est employé pour déterminer où les rayons projetés à travers la 
        scène intersectent avec les objets de l'environnement, typiquement et 
        dans notre cas représenté par une grille de cellules. Le principe 
        fondamental de DDA repose sur l'itération linéaire. Plutôt que de 
        calculer chaque point le long d'une demie droite en utilisant des formules 
        de géométrie directe, ce qui pourrait être coûteux en termes de performances, 
        DDA avance par petits incr ́ements. Cela signifie que pour chaque
        pas sur l'axe le plus dominant (x ou y), DDA calcule l'emplacement 
        correspondant sur l'autre axe en ajoutant un incrément constant. 
        Cela se traduit par un parcours régulier et efficace le long de la 
        ligne en faisant des sauts à chaque intersection entre la demie 
        droite et la grille 2D. Dans le raycasting, DDA est utilisé pour 
        trouver rapidement et précisément les intersections entre les rayons 
        et les murs
    \end{block}    
\end{frame}

\begin{frame}
    \frametitle{Différentes stratégie pour un rendu 3D \\
                \small Approche moderne}           
    \begin{block}{}
        
    \end{block}    
\end{frame}

\subsection{Construction de mur}

\begin{frame}
    \frametitle{Construction de mur}
    \begin{block}{}
    \end{block}
\end{frame}

\subsection{Les collisions}

\begin{frame}
    \frametitle{Les collisions}
    \begin{block}{}
    \end{block}
\end{frame}

\subsection{Les portails}

\begin{frame}
    \frametitle{Les portails}
    \begin{block}{title}
        
    \end{block}
\end{frame}

\subsection{Les renders}

\begin{frame}
    \frametitle{Les renders}
    \begin{block}{title}
        
    \end{block}
\end{frame}

\section{Conclusion}

\begin{frame}
    \frametitle{Conclusion}
    \begin{block}{}
        \centering
        Merci de votre attention.
    \end{block}
\end{frame}

\section*{Questions}

\begin{frame}
    \frametitle{Questions}
    \begin{block}{}
        \centering
        Avez-vous des questions ?
    \end{block}
\end{frame}

\end{document}