% Document principal LaTeX pour compte-rendu
\documentclass[12pt]{report}
\usepackage{BidoTexCourses}
\usepackage{hyperref}


\title{Compte rendu projet Portal 0.0}
\author{BIDAULT Matthieu, BADSTÜBER Elian, FOCHEUX Vital}
\date{\today}

\renewcommand{\contentsname}{
	Table des matières
}

\begin{document}

\maketitle

\section*{Remerciements}

\paragraph{}
Nous tenons à remercier notre enseignant de projet, M. \textsc{Bernard} 
qui a su nous guider et nous conseiller tout au long de ce projet.


\tableofcontents

\section{Introduction}

\paragraph{}

Dans le cadre du projet semestriel de troisième année de licence
informatique à l'université de Franche-Comté, nous proposons de coder
une version très simplifiée du jeu \href{https://fr.wikipedia.org/wiki/Portal_(jeu_vid%C3%A9o)}{Portal}
qui sera afficher grâce à un moteur de type Raycaster à la façon du jeu
\href{https://fr.wikipedia.org/wiki/Wolfenstein_3D}{Wolfenstein 3D}.
Il a pour but de nous faire découvrir le monde du développement de jeux 
vidéo en nous faisant réaliser un jeu vidéo en C++ avec la bibliothèque 
GF.

\clearpage

\section{Besoins et objectifs du projet}
\subsection{Contexte}

\paragraph{Pour les graphismes}
\paragraph{}
Dans le début des années 90, la société \href{https://fr.wikipedia.org/wiki/Id_Software}{Id Software} a commencé à faire
des expérimentations dans le domaine des graphismes 3D, qui était jusque-là, presque uniquement utilisés dans les simulateurs de vols
comme \href{https://fr.wikipedia.org/wiki/Wing_Commander_(jeu_vid%C3%A9o)}{Wing Commander} ou \href{https://en.wikipedia.org/wiki/Knights_of_the_Sky}{Knights of the Sky}
qui sont tous les deux sortis en 1990. A cause de la puissance des ordinateurs de l'époque, il était difficile de faire des jeux
d'action en 3D et rapide. C'est pourquoi \href{https://fr.wikipedia.org/wiki/John_Carmack}{John Carmack} a eu l'idée de faire un moteur
en créant les graphismes grâce à la technique du \href{https://fr.wikipedia.org/wiki/Raycasting}{raycasting}. Cette technique consiste
à ne calculer que les surfaces visibles par le joueur plutôt que toutes celles qui l'entourent. Au bout de six semaines il abouti 
à un moteur 3D qui utilise des sprites en 2D pour représenter les entités du jeu. Ce moteur a été utilisé pour le jeu 
\href{https://fr.wikipedia.org/wiki/Hovertank_3D}{Hovertank 3D} publié en avril 1991.
En automne 1991, alors que \href{https://fr.wikipedia.org/wiki/John_Carmack}{John Carmack} et \href{https://fr.wikipedia.org/wiki/John_Romero}{John Romero}
terminent le moteur de \href{https://en.wikipedia.org/wiki/Commander_Keen_in_Goodbye,_Galaxy}{Commander Keen in Goodbye, Galaxy}, 
John Carmack entend parler d'un jeu de rôle développé par \href{https://fr.wikipedia.org/wiki/Looking_Glass_Studios}{Blue Sky Productions}
dénommé \href{https://fr.wikipedia.org/wiki/Ultima_Underworld}{Ultima Underworld}. Leurs créateurs ont réussi à créer un moteur 
permmettant de faire des graphismes 3D et avec des textures, sans les restrictions du moteur de Hovertank 3D.
John Carmack décide alors d'améliorer son moteur pour, comme celui d'Ultima Underworld, pourra bénéficier
d'un mapping de textures sans pour autant en sacrifier les performances.
Après six semaines de travail, il abouti a ce nouveau moteur 3D qui est utilisé pour créer
un nouveau jeu pour Softdisk, \href{https://fr.wikipedia.org/wiki/Catacomb_3D}{Catacomb 3D}, publié en novembre 1991.
Après la découverte de ce dernier jeu, \href{https://fr.wikipedia.org/wiki/Scott_Miller_(programmeur)}{Scott Miller} de la socitété
Apogee pousse l'équipe de développement à créer un jeu d'action en 3D en shareware. C'est ainsi que le début du projet de
\href{https://fr.wikipedia.org/wiki/Wolfenstein_3D}{Wolfenstein 3D}, qui est un remake de \href{https://fr.wikipedia.org/wiki/Castle_Wolfenstein}{Castle Wolfenstein}
en 3D, a commencé. Publié le 5 mai 1992 sur PC, ce jeu a été un succès et a été le premier jeu de tir à la première personne qui a popularisé le genre.
Il a initié les jeux de tir 2.5D et a été le précurseur de \href{https://fr.wikipedia.org/wiki/Doom_(jeu_vid%C3%A9o,_1993)}{Doom} et
\href{https://fr.wikipedia.org/wiki/Quake}{Quake}.
% Dans le début des années 90, un projet de jeu vidéo nommé \href{https://fr.wikipedia.org/wiki/Wolfenstein_3D}{Wolfenstein 3D}
% a commencé à voir le jour. Il s'agit d'un jeu de tir à la première personne
% qui a été développé par \href{https://fr.wikipedia.org/wiki/Id_Software}{Id Software} et publié par
% \href{https://fr.wikipedia.org/wiki/Apogee_Software}{Apogee Software} le 5 mai 1992 sur PC.
% Ce jeu a été un succès et a été le premier jeu de tir à la première personne qui a popularisé le genre.
% Il a initié les jeux de tir 2.5D et a été le précurseur de \href{https://fr.wikipedia.org/wiki/Doom_(jeu_vid%C3%A9o,_1993)}{Doom} et
% \href{https://fr.wikipedia.org/wiki/Quake}{Quake}.

\paragraph{Pour le système de jeu}
\paragraph{}
En 2007, la société \href{https://fr.wikipedia.org/wiki/Valve_Corporation}{Valve Corporation} a publié un jeu vidéo nommé
\href{https://fr.wikipedia.org/wiki/Portal_(jeu_vid%C3%A9o)}{Portal} qui est un 
jeu de réflexion à la première personne. Le joueur possède un générateur de portail qui lui permet de créer deux portails distincts,
l'un orange, l'autre blue sur des surfaces planes. Le joueur peut passer à travers ces portails et ainsi se déplacer dans l'espace
tout en conservant sa vitesse de déplacement. Le but du jeu est de résoudre des énigmes en utilisant les portails et d'atteindre la sortie.

\clearpage

\subsection{Motivations}

\paragraph{}

L'une des motivations principales de ce projet est de réaliser un jeu vidéo
avec graphique comme à l'époque de \href{https://fr.wikipedia.org/wiki/Wolfenstein_3D}{Wolfenstein 3D} 
mais avec des comcepts de jeux plus récentes et qui plus est pourrai être
un préquel de \href{https://fr.wikipedia.org/wiki/Portal_(jeu_vid%C3%A9o)}{Portal}.


\subsection{Objectif et contraintes}

\paragraph{}
Les principaux objectifs de ce projet sont :
\begin{itemize}
	\item Réaliser un jeu vidéo en C++ avec la bibliothèque GF
	\item Implémenter un moteur de type Raycaster
	\item Implémenter un système de portail
	\item Implémenter un système de collision
	\item Offir une expérience de jeu simple et agréable à jouer.
\end{itemize}

\paragraph{}
Mais avec des contraintes :
\begin{itemize}
	\item L'apprentissage d'un langage de programmation nouveau pour nous
	\item L'apprentissage d'une bibliothèque de programmation nouvelle pour nous
	\item L'apprentissage de la programmation d'un moteur de type Raycaster
	\item L'apprentissage de la programmation d'un système de portail
	\item L'apprentissage de la programmation d'un système de collision
	\item L'apprentissage de la programmation d'un système de jeu
\end{itemize}

\section{Gestion du projet}
\subsection{L'équipe}
\subsection{Planification et outils de gestion}

\paragraph{}
Pour la gestion du projet nous avons utilisé le site 
\href{https://github.com/}{github} qui est un outil de gestion 
de projet en ligne.

\subsection{Répartition des tâches}

\section{Développement}
\subsection{Différentes stratégie}
\subsection{Programmer en C++}
\subsection{Apprendre à utiliser la bibliothèque GF}

\section{Bilan du projet}
\subsection{Résultats obtenus}
\subsection{Apports technique}
\subsection{Apports personnels}

\section{Perspectives}
\subsection{Améliorations possibles}
\subsection{Nouvelles fonctionnalités}
\subsection{Un plus dans nos CV}

\section{Bibliographie}


\end{document}