% Document principal LaTeX pour compte-rendu
\documentclass[12pt]{report}
\usepackage{BidoTexCourses}
\usepackage{hyperref}


\title{Compte rendu projet Portal 0.0}
\author{BIDAULT Matthieu, BADSTÜBER Elian, FOCHEUX Vital}
\date{\today}

\renewcommand{\contentsname}{
	Table des matières
}

\begin{document}

\maketitle

\section*{Remerciements}

\paragraph{}
Nous tenons à remercier notre enseignant de projet, M. \textsc{Bernard} 
qui a su nous guider et nous conseiller tout au long de ce projet.


\tableofcontents

\section{Introduction}

\paragraph{}

Dans le cadre du projet semestriel de troisième année de licence
informatique à l'université de Franche-Comté, nous proposons de coder
une version très simplifiée du jeu \href{https://fr.wikipedia.org/wiki/Portal_(jeu_vid%C3%A9o)}{Portal}
qui sera afficher grâce à un moteur de type Raycaster à la façon du jeu
\href{https://fr.wikipedia.org/wiki/Wolfenstein_3D}{Wolfenstein 3D}.
Il a pour but de nous faire découvrir le monde du développement de jeux 
vidéo en nous faisant réaliser un jeu vidéo en C++ avec la bibliothèque 
GF.

\section{Besoins et objectifs du projet}
\subsection{Contexte}
\subsection{Motivations}

\paragraph{}

L'une des motivations principales de ce projet est de réaliser un jeu vidéo
avec graphique comme à l'époque de \href{https://fr.wikipedia.org/wiki/Wolfenstein_3D}{Wolfenstein 3D} 
mais avec des comcepts de jeux plus récentes et qui plus est pourrai être
un préquel de \href{https://fr.wikipedia.org/wiki/Portal_(jeu_vid%C3%A9o)}{Portal}.


\subsection{Objectif et contraintes}

\paragraph{}
Les principaux objectifs de ce projet sont :
\begin{itemize}
	\item Réaliser un jeu vidéo en C++ avec la bibliothèque GF
	\item Implémenter un moteur de type Raycaster
	\item Implémenter un système de portail
	\item Implémenter un système de collision
	\item Offir une expérience de jeu simple et agréable à jouer.
\end{itemize}

\paragraph{}
Mais avec des contraintes :
\begin{itemize}
	\item L'apprentissage d'un langage de programmation nouveau pour nous
	\item L'apprentissage d'une bibliothèque de programmation nouvelle pour nous
	\item L'apprentissage de la programmation d'un moteur de type Raycaster
	\item L'apprentissage de la programmation d'un système de portail
	\item L'apprentissage de la programmation d'un système de collision
	\item L'apprentissage de la programmation d'un système de jeu
\end{itemize}

\section{Gestion du projet}
\subsection{L'équipe}
\subsection{Planification et outils de gestion}

\paragraph{}
Pour la gestion du projet nous avons utilisé le site 
\href{https://github.com/}{github} qui est un outil de gestion 
de projet en ligne.

\subsection{Répartition des tâches}

\section{Développement}
\subsection{Différentes stratégie}
\subsection{Programmer en C++}
\subsection{Apprendre à utiliser la bibliothèque GF}

\section{Bilan du projet}
\subsection{Résultats obtenus}
\subsection{Apports technique}
\subsection{Apports personnels}

\section{Perspectives}
\subsection{Améliorations possibles}
\subsection{Nouvelles fonctionnalités}
\subsection{Un plus dans nos CV}

\section{Bibliographie}


\end{document}